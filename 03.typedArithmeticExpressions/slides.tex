\documentclass{beamer}

\usepackage{busproofs}

\usepackage{amsmath}

\usefonttheme[onlymath]{serif}

\usepackage{ifxetex,ifluatex}
\usepackage{etoolbox}
% \usepackage[svgnames]{xcolor}

\usepackage{tikz}

\usepackage{framed}

% conditional for xetex or luatex
\newif\ifxetexorluatex
\ifxetex
  \xetexorluatextrue
\else
  \ifluatex
    \xetexorluatextrue
  \else
    \xetexorluatexfalse
  \fi
\fi
%
\ifxetexorluatex%
  \usepackage{fontspec}
  \usepackage{libertine} % or use \setmainfont to choose any font on your system
  \newfontfamily\quotefont[Ligatures=TeX]{Linux Libertine O} % selects Libertine as the quote font
\else
  \usepackage[utf8]{inputenc}
  \usepackage[T1]{fontenc}
  \usepackage{libertine} % or any other font package
  \newcommand*\quotefont{\fontfamily{LinuxLibertineT-LF}} % selects Libertine as the quote font
\fi

\newcommand*\quotesize{60} % if quote size changes, need a way to make shifts relative
% Make commands for the quotes
\newcommand*{\openquote}
   {\tikz[remember picture,overlay,xshift=-4ex,yshift=-2.5ex]
   \node (OQ) {\quotefont\fontsize{\quotesize}{\quotesize}\selectfont``};\kern0pt}

\newcommand*{\closequote}[1]
  {\tikz[remember picture,overlay,xshift=4ex,yshift={#1}]
   \node (CQ) {\quotefont\fontsize{\quotesize}{\quotesize}\selectfont''};}

% select a colour for the shading
\colorlet{shadecolor}{lightgray}

\newcommand*\shadedauthorformat{\emph} % define format for the author argument

% Now a command to allow left, right and centre alignment of the author
\newcommand*\authoralign[1]{%
  \if#1l
    \def\authorfill{}\def\quotefill{\hfill}
  \else
    \if#1r
      \def\authorfill{\hfill}\def\quotefill{}
    \else
      \if#1c
        \gdef\authorfill{\hfill}\def\quotefill{\hfill}
      \else\typeout{Invalid option}
      \fi
    \fi
  \fi}
% wrap everything in its own environment which takes one argument (author) and one optional argument
% specifying the alignment [l, r or c]
%
\newenvironment{shadequote}[2][l]%
{\authoralign{#1}
\ifblank{#2}
   {\def\shadequoteauthor{}\def\yshift{-2ex}\def\quotefill{\hfill}}
   {\def\shadequoteauthor{\par\authorfill\shadedauthorformat{#2}}\def\yshift{2ex}}
\begin{snugshade}\begin{quote}\openquote}
{\shadequoteauthor\quotefill\closequote{\yshift}\end{quote}\end{snugshade}}


\title{Types and Programming Languages} 
\subtitle{Typed Arithmetic Expressions}

\author{Rodrigo Bonif\'{a}cio}
\date{2017/08}

\begin{document}

\begin{frame}
\titlepage
\end{frame}

\begin{frame}

Let's just remember our {\color{blue}Arithmetic Expression}
language. 

\end{frame}

\begin{frame}
\frametitle{Syntactic Forms} 

\begin{itemize}
\item boolean constants: \texttt{true}, \texttt{false}
\item conditional expressions
\item numeric constant: \texttt{zero}
\item arithmetic operators: \texttt{succ}, \texttt{pred}
\item testing operation: \texttt{isZero} 
\end{itemize}
\end{frame}

\begin{frame}[fragile]
\frametitle{Grammar} 

\begin{verbatim}
 Exp ::= true
         false 
         if Exp then Exp else Exp
         0
         succ Exp
         pred Exp
         iszero Exp
\end{verbatim}
\end{frame}

\begin{frame}[fragile]
In this language, the results of evaluation
are terms of a particularly simple form: they will 
either evaluate to a value 

\begin{verbatim}
 v ::= true 
     | false 
     | nv

 nv ::= 0
      | succ nv 
\end{verbatim}

or else get {\color{blue}stuck} at some stage (by 
reaching a term like \texttt{pred false} for which 
no evaluation rule applies). 
\end{frame}

\begin{frame}
\begin{itemize}
\item Our goal is to design means to tell, without evaluating 
a term, that its evaluation will definitely not get stuck, 
by assigning a term $t$ to a given type $T$.  \pause 

\item {\color{blue}\ldots without evaluating a term} means that 
we shoud assign a type to a term {\color{blue}statically}, 
and thus we will follow a conservative approach, rejecting 
a program of the form \texttt{if true then 0 else false}---even though 
it won't get stuck. 
\end{itemize}
\end{frame}

\begin{frame}[fragile]
\frametitle{The Type Relation} 

\begin{block}{New syntactic forms}
\begin{verbatim} 
 T ::= Bool            type of booleans
     | Nat      type of natural numbers       
\end{verbatim}
\end{block} \pause 

\begin{prooftree}
\AxiomC{}
\RightLabel{\quad (T-Zero)}
\UnaryInfC{$0\ : Nat$}
\end{prooftree}

\begin{prooftree}
\AxiomC{}
\RightLabel{\quad (T-True)}
\UnaryInfC{$true\ : Bool$}
\end{prooftree}

\begin{prooftree}
\AxiomC{}
\RightLabel{\quad (T-False)}
\UnaryInfC{$false\ : Bool$}
\end{prooftree}

\end{frame}

\begin{frame} 

\begin{prooftree}
\AxiomC{$t_1 : Bool$} 
\AxiomC{$t_2 : T$}
\AxiomC{$t_3 : T$}
\RightLabel{\quad (T-If)}
\TrinaryInfC{$If\ t_1\ Then\ t_2\ Else\ t_3\ : T$}
\end{prooftree}

\begin{prooftree}
\AxiomC{$t_1 : Nat$}
\RightLabel{\quad (T-Succ)}
\UnaryInfC{$succ\ t_1: Nat$}
\end{prooftree} 

\begin{prooftree}
\AxiomC{$t_1 : Nat$}
\RightLabel{\quad (T-Pred)}
\UnaryInfC{$pred\ t_1: Nat$}
\end{prooftree} 

\begin{prooftree}
\AxiomC{$t_1 : Nat$}
\RightLabel{\quad (T-IsZero)}
\UnaryInfC{$iszero\ t_1: Bool$}
\end{prooftree} 

\end{frame}

\begin{frame}
\begin{block}{Type Soundness (safety)}
\begin{itemize}
 \item Progress 
 \item Preservation 
\end{itemize}
\end{block}

\pause This is one of the most important 
concepts of our course. 
\end{frame}

\begin{frame}
\huge{Haskell implementation}
\end{frame}

\begin{frame}
 \titlepage
\end{frame}
\end{document}
